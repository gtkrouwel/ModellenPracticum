\documentclass[11pt]{article}
\usepackage[margin=25mm]{geometry}
\usepackage{arev}
\usepackage[T1]{fontenc}
\usepackage[UKenglish]{babel}
\usepackage{amsmath}

\title{Reliably modelling the temperature of underground electricity cables}
\author{
Arthur Al-Sett \\
Joie Geurts \\
Pieter Cas Kolijn \\
Gabriël Krouwel
\and
Erik Mulders \\
Davy Peters \\
Davy Westra
}

\begin{document}
\maketitle
%\toc  % TODO uncomment when document is of substantiable size.

\section{Introduction}
The use of renewable energy is increasing. This electricity is produced in many locations rather than in a central one. As a result, more % TODO write more text.

\section{Goals and Questions}  % TODO heading could be better.
\begin{enumerate}
    \item Deduce the temperature of underground power cables using propagation time measurements by the Smart Cable Guard system, and the linear relation
    \begin{equation} \label{eq:CableTemperaturePropagationTimeModel}
        T_{\text{cable}}(t) = \alpha_0 + \alpha_1 P(t) + \epsilon(t),
    \end{equation}
    where $\alpha_0$ and $\alpha_1$ are constants and at time $t$, $T_{\text{cable}}(t)$ is the insulation temperature of the cable, $P(t)$ is the measured propagation time, and $\epsilon(t)$ is the error.

    % Question 1 from Rieken's <README.md>.
    \item When little electricity runs through a certain cable, the cable temperature is close to the soil temperature and therefore the correlation between the soil temperature and the propagation time is high: see \eqref{eq:CableTemperaturePropagationTimeModel}.
    How can we quickly recognize that this is the case, without knowledge of how much electricity runs through a certain cable at a given time?
    
    % TODO put rest of the questions.
    \item 

\end{enumerate}

\end{document}